\documentclass[14pt]{beamer}
\usepackage[utf8]{inputenc}
\usepackage[T1]{fontenc}
\usepackage[portuguese]{babel}
\usepackage{amsmath}
\usepackage{amsfonts}
\usepackage{amssymb}
\usepackage{graphicx}
\usetheme{CambridgeUS}
\begin{document}
	\author{Alexandre Rogério}
	\title{Exibindo o conteúdo pausadamente}
	%\subtitle{}
	%\logo{}
	%\institute{}
	%\date{}
	%\subject{}
	%\setbeamercovered{transparent}
	%\setbeamertemplate{navigation symbols}{}
	\begin{frame}[plain]
	\maketitle
\end{frame}

\begin{frame}
\frametitle{Exemplo 1}
\transboxin
\begin{block}{Bloco}
	Quero esse conteúdo aparecendo antes. \pause
	Quero esse conteúdo aparecendo depois.
\end{block}
\end{frame}

\begin{frame}
\frametitle{Exemplo 2}
\begin{block}{Exemplo de bloco}
	\begin{enumerate}[<+->]
		\item primeiro item.
		\item segundo item.
		\item terceiro item.
	\end{enumerate}
\end{block}
\end{frame}

\begin{frame}
\frametitle{Exemplo 3}
\begin{block}{Exemplo de bloco}
	\begin{itemize}
		\only<1-3>{\item Equação de primeiro grau:}
		\only<3>{\begin{equation}
			ax + b = 0
		\end{equation}}
		\uncover<2->{\item Equação de segundo grau:}
		\only<4->{\begin{equation}
		ax^2 + bx + c = 0
		\end{equation}}
	\end{itemize}
\end{block}
\end{frame}
\end{document}