\documentclass{article}

\usepackage[utf8]{inputenc}
\usepackage[brazil]{babel}
\usepackage{indentfirst}
\usepackage[a4paper, left=1cm, right=2cm, top=2cm, bottom=3cm]{geometry}
\usepackage{graphicx}
\usepackage{float}
\usepackage{multirow}
\usepackage{tabularx}
\usepackage{amsmath} % Modo matemático

\renewcommand{\sin}{\mathrm{sen\hspace{0.5mm}}}
\renewcommand{\tan}{\mathrm{tg\hspace{0.5mm}}}

\begin{document}	
	\title{\textbf{{\Huge Modo matemático - erros comuns}}}
	\author{Alexandre Rogério}
	%\date{}
	\maketitle
	\thispagestyle{empty} % Oculta a numeração da página
	\newpage
	
	\setcounter{page}{1}
	\pagenumbering{Roman}
	\tableofcontents
	\newpage
	
	\listoffigures
	\newpage
	
	\listoftables
	\newpage
	
	\setcounter{page}{1}
	\pagenumbering{arabic}
	
	\section{Modo matemático}
	Esta é a equação de segundo grau: $ ax^2 + bx + c = 0 $ . A solução é:
	\begin{equation*}
		\Delta = b^2 - 4ac
	\end{equation*}
	\begin{equation*}
		x = \frac{-b \pm \sqrt{\Delta}}{2a}
	\end{equation*}

	\begin{equation*}
		\begin{array}{cc}
			x_1 = \dfrac{-b + \sqrt{\Delta}}{2a}	&
			x_2 = \dfrac{-b - \sqrt{\Delta}}{2a}
		\end{array}
	\end{equation*}

	\section{Matriz}
	\begin{equation*}
		A = \begin{bmatrix}
			1 & 0 & 0 \\
			0 & 1 & 0 \\
			0 & 0 & 1
		\end{bmatrix}
	\end{equation*}

	\section{Comandos}
	\begin{equation*}
		\sin{2x}
	\end{equation*}

	\begin{equation*}
		\tan{2x}
	\end{equation*}

	\section{Utilidades}
	\begin{equation*}
		\left(\frac{2a}{3b}\right)
	\end{equation*}
	\begin{equation*}
		\{2a\}
	\end{equation*}
	\begin{equation*}
		100\%
	\end{equation*}
	\begin{equation*}
		x_{12}^2
	\end{equation*}
\end{document}